\documentclass[11pt]{article}
\usepackage[margin=1in]{geometry}
\usepackage[T1]{fontenc}
\usepackage[polish]{babel}
\usepackage[utf8]{inputenc}
\usepackage{lmodern}
\usepackage{hyperref}

\selectlanguage{polish}

\title{Zasady krytycznego myślenia - zadanie 1}
\author{Wojciech Kołowski}
\date{kwiecień 2018}

\begin{document}
	\maketitle
	
	\par W ramach zadania przeanalizuję tekst zatytułowany ''Dochód podstawowy'', którego autorem jest Łukasz Żołądek. Tekst został opublikowany przez Biuro Analiz Sejmowych 14 września 2016 i jest łatwo dostępny w Internecie.
	
	\section{Wstępna ocena stronniczości}
	\begin{itemize}
		\item Nie ma zauważalnego na pierwszy rzut oka konfliktu interesów.
		\item Artykuł jest pisany w miarę neutralnym językiem (nie wliczając niektórych cytatów).
		\item Wydaje się, że w tej sprawie nie ma nacisków politycznych - jak zauważa autor, temat jest na marginesie debaty publicznej.
		\item Artykuł jest wybiórczy w prezentacji. Przypis nr 12 głosi ''Artykuł przedstawia argumentację zwolenników dochodu podstawowego''. Argumenty przeciwników zostały pominięte lub jedynie wspomniane i omówione jednym zdaniem.
		\item Autor jest doktorantem socjologii, co może powodować, że jest zainteresowany wprowadzeniem dochodu podstawowego dla uzyskania korzyści materialnych.
		\item Na bazie powyższych przesłanek można uznać, że autor jest zwolennikiem wprowadzenia dochodu podstawowego i że zarówno autor, jak i artykuł są stronniczy.
	\end{itemize}
	
	\section{Klasyfikacja fragmentów tekstu}
		\par Tekst składa się z czterech części:
		\begin{itemize}
			\item Na wstępie autor podaje definicję dochodu podstawowego oraz przytacza ówczesne wydarzenia polityczno-społeczne powiązane z tym tematem.
			\item W sekcji ''Gospodarka, rynek pracy i ubezpieczenie społeczne: diagnoza sytuacji'' autor przedstawia, jak jego zdaniem wyglądają współczesne gospodarki, systemy zabezpieczeń społecznych i w jaki sposób stan współczesny wynika ze stanu historycznego. Przytacza też garść statystyk i opisuje też ogólną sytuacja na świecie w sprawach mogących mieć związek z dochodem podstawowym.
			\item W sekcji ''Dochód podstawowy: argumenty i propozycje'' autor przedstawia koncepcję dochodu podstawowego nieco szerzej. Analizuje potencjalne skutki wprowadzenia różnych jego odmian oraz porównuje je ze skutkami już istniejących programów pomocy społecznej. Przytacza też przykłady różnych rządowych programów i eksperymentów, które swym kształtem przypominają koncepcję dochodu podstawowego.
			\item Sekcja ''W stronę świadczeń (bardziej) powszechnych?'' to bardzo skromne i mało znaczące podsumowanie.
		\end{itemize}
	
	\section{Dygresje i wątki uboczne}
		\par Tekst zawiera jeden akapit, który zdaje się być odwracaniem uwagi. Autor klasyfikuje w nim zwolenników na linii lewica-prawica oraz wspomina o mocno odmiennej koncepcji ''negatywnego podatku dochodowego''. Nie wydaje się jednak, żeby ten akapit wprowadzał jakiś straszny zamęt lub skrywał propagandę.
		
	\section{Referowanie faktów i stanowisk}
		\par Jak wspomniano na wstępie, autor jest wybiórczy i stronniczy, choć na jego korzyść przemawia fakt, że wprost się do tego przyznaje. Tekst jest krótki, więc fakty i stanowiska referowane są pobieżnie, ale ich rzetelność zdaje się być w porządku.
		\par Przytoczone statystyki są nie do końca wiarygodne: mimo iż autor cytuje źródła, to jednym z nich jest książka ''Kapitał w XXI wieku'', która była wielokrotnie krytykowana za (rzekomą) nierzetelność w podawaniu danych statystycznych (patrz \url{https://en.wikipedia.org/wiki/Capital_in_the_Twenty-First_Century#Allegation_of_data_errors}). Nie podjąłem się dokładnego zweryfikowania przytaczanych danych, gdyż byłoby to bardzo czasochłonne.
		\par Interpretację danych autor zostawił czytelnikowi, co z konieczności będzie prowadzić do błędów i nieporozumień. Przykładem manipulacji jest stwierdzenie ``(\dots) najlepiej żyje się w społeczeństwach o umiarkowanych nierównościach'', które ma sugerować czytelnikowi, że występuje między tymi dwoma statystykami (z których jedna, poziom życia, jest moco nieprecyzyjna) związek przyczynowo-skutkowy.
		
	\section{Główna teza - o co autorowi chodzi?}
		\begin{quote}
			``Dochód podstawowy (\dots) jest czymś więcej - stanowi szeroko zakrojoną propozycję mającą prowadzić do zmiany społecznej, poprawiając sytuację jednostek, społeczeństw i gospodarki; proponuje zmianę sposobu myślenia o jednostce i wspólnocie.''
		\end{quote}
		\par Powyższy cytat dobrze obrazuje pogląd autora na dochód podstawowy. Głównym celem autora jest przekonanie czytelnika do podzielenia tego poglądu.
		\par Autor próbuje też nakłonić czytelnika, by ten uwierzył, że wprowadzenie dochodu podstawowego zmniejszy biedę i nierówności społeczne, które autor przedstawia w negatywnym świetle, a których przyczyn upatruje w deregulacji i liberalizacji.
		\par Ogólniej celem autora jest zaszczepienie w czytelniku pozytywnego nastawienia do dochodu podstawowego oraz zachęcenie go do samodzielnego zapoznania się z tekstami go promującymi (ale nie krytykującymi lub neutralnymi, o czym świadczy stronniczość prezentacji).
		\par Jeszcze ogólniej celem autora jest przekonanie czytelnika do swojej socjalistycznej wizji świata (o czym świadczą nazwiska cytowanych i tytuły ich książek: Guy Standing, ''Prekariat'', Thomas Piketty, ''Kapitał w XXI wieku''). Argumentem za tym przemawiającym jest fakt, że autor nie wymienia innych znanych osób popierających ideę dochodu podstawowego, lecz posiadających zupełnie inny pogląd na większość innych spraw (jak np. noblista Friedrich von Hayek).
	
	\section{Niejawne przesłanki i niejawne wnioski}
		\par Tekst naszpikowany jest niejawnymi przesłankami, mającymi na celu wtłoczyć myślenie czytelnika w pewne ramy. Wymienione zostaną one w miarę możliwości w porządku pojawiania się między wierszami tekstu. Dla krótkości ograniczę się jedynie do tych z sekcji ``Gospodarka, rynek pracy i ubezpieczenie społeczne: diagnoza sytuacji''.
		\begin{itemize}
			\item Definiując dochód podstawowy, autor pisze ``Jest to świadczenie (\dots) przeznaczone dla wszystkich obywateli (i ewentualnie innych stałych mieszkańców) (\dots)''. W ten sposób niejawnie zakłada, że kwestia tego, czy imigranci powinni dostawać dochód podstawowy, jest nieistotna, podobnie jak kwestia tego, jakie prawa obowiązują w zakresie obywatelstwa. Przytoczona definicja pomija też kwestię tego, czy dochód podstawowy powinien przysługiwać dzieciom. Kwestie te są istotne, gdyż mogą powodować diametralnie różne skutki: napływ imigrantów lub wzmożone rozmnażanie się patologii.
			\item Autor przyjmuje socjalistyczną retorykę i punkt widzenia. Używa on wątpliwych pojęć takich jak ``usługi społeczne'' (czym różnią się one od zwykłych usług?). Twierdzi, że ``skutecznie łączono gospodarkę rynkową z interwencjonizmem państwowym'', co jest sprzecznością, gdyż rynek i interwencja państwa są swoimi przeciwieństwami. Przedstawia on ``deregulację'' i ``liberalizację'' jako nieskuteczne odpowiedzi na kryzys lat 70. Pisze, że ubezpiezenia społeczne ``są coraz mniej skuteczne'' zakładając tym samym, że kiedykolwiek były. Twierdzi też, że ``Redystrybucja (\dots) przestała działać'' zakładając tym samym, że kiedykolwiek działała.
			\item Autor twierdzi też, że na Zachodzie miał miejsce cud gospodarczy i przypisuje go ``świetności \textit{welfare state}''. W rzeczywistości to, co działo się po wojnie na Zachodzie nie było żadnym cudem, lecz konsekwentnym zastosowaniem idei wolnorynkowych, a więc dokładnie przeciwnych do tych zasugerowanych przez autora, i to tylko w niektórych krajach, takich jak Niemcy Zachodnie czy Hiszpania (inne krajach, jak Holandia czy Szwecja, były już wówczas bogate).
			\item Autor twierdzi, że ``(\dots) pojawiłą się kategoria biednych pracujących'', zakładając tym samym niejawnie, że kiedyś takiej klasy nie było. Jest to dość mocno niespójne z socjalistyczną wizją dziewiętnastowiecznego kapitalizmu, w którym robotnicy byli biedakami. Jest również oczywiste, że np. przed wojną ludzie pracujący byli biedniejsi niż obecni pracujący, gdyż nie mieli lodówek, samochodów, telefonów, komputerów etc.
			\item Pisząc ``Umowy o pracę są wypierane przez tzw. elastyczne formy zatrudnienia, co oznacza, że pracujący nie korzystają z praw pracowniczych.'' autor zakłada niejawnie, że korzystanie z praw pracowniczych jest korzystne dla pracowników, co nie musi być prawdą - może to powodować wzrost kosztów po stronie pracodawcy, a zatem wzrost cen i spadek siły nabywczej wszystkich pracowników.
			\item Autor narzeka na ``niestabilność zatrudnienia'' zakładając tym samym, że istnieje coś takiego jak stabilne zatrudnienie. W gospodarce rynkowej jednak tak nie jest - rynek jest dynamiczny, gdyż przedsiębiorcy muszą stale dostosowywać produkcję do woli konsumentów i innych realiów.
			\item Autor narzeka również na ``wykonywanie pracy poniżej poziomu wykształcenia'' zakładając tym samym, że każdemu przysługuje prawo do wykonywania pracy odpowiadającej swojemu wykształceniu, co jest nonsensem.
			\item Autor niejawnie zakłada, że nierówności społeczne są niekorzystne, co jest nieprawdą. Nierówności muszą istnieć, aby motywować ludzi do większej produktywności i przedsiębiorczości.
			\item Autor twierdzi, że ``skrajnie nierówności podkopują składaną przez wolny rynek obietnicę, że to praca i talent zapewniają sukces'', zakładając tym samym niejawnie, że zarządzanie majątkiem i dysponowanie kapitałem nie wymaga pracy i talentu, co jest nieprawdą.
			\item Autor przytacza podział przez zwolenników DP wolności na ``formalną'' i jakąś inną, którą na nasze potrzeby nazwiemy realną. Pierwsza odpowiada koncepcji wolności negatywnej (``wolności od''), druga zaś koncepcji wolności pozytywnej, czyli ``wolności do''. Druga z nich to rzecz jasna bzdura. Autor nie krytykuje ani nie obala tego podziału, akceptuje więc go.
			\item Autor posługuje się też pojęciem sprawiedliwości bez precyzowania go, niejawnie przyjmując tym samym, że czytelnik podziela jego koncepcję sprawiedliwości.
		\end{itemize}
		
		\begin{quote}
			``Zwolennicy dochodu podstawowego dochodzą zatem do wniosku, który można ująć następująco: państwo opiekuńcze z tradycyjnymi zabezpieczeniami społecznymi wywodzi się z epoki uprzemysłowienia i stanowiło opdowiedź na wyzwania swoich czasów. Natomiast dziś, w epoce postindustrialnej, stoimy przed innymi wyzwaniami, które wymagają nowych pomysłów i nowoczesnej polityki społecznej.''
		\end{quote}
		\par W powyższym fragmencie autor podaje swoje wnioski niejawnie na dwóch poziomach:
		\begin{itemize}
			\item Po pierwsze, podaje swoje wnioski jako wnioski zwolenników dochodu podstawowego.
			\item Po drugie, ostatecznym niejawnym wnioskiem płynącym z cytatu jest stwierdzenie, że mamy problemy i że ich rozwiązaniem jest wprowadzenie dochodu podstawowego, a zatem także, że dochód podstawowy jest dobry, korzystny, ``świeży'', nowoczesny etc.
		\end{itemize}
	
	\section{Diagram argumentacji}
		\par Struktura argumentacji jest dość prosta:
		\begin{enumerate}
			\item Mamy problemy.
			\item Dochód podstawowy rozwiąże nasze problemy.
			\item Ogólne argumenty za (2). Dochód podstawowy to:
			\begin{enumerate}
				\item Reforma państwa opiekuńczego (argument podany jako stanowisko Guya Standinga).
				\item ``(\dots) walka przeciw tyranii szefów, mężów i biurokratów.'' (argument podany jako stanowisko Philippe'a Van Parijsa).
				\item Walka o ``wolność realną''.
				\item Sprawiedliwie urządzone społeczeństwo.
				\item Godność, autonomia i emancypacja jednostki.
				\item Konieczność mająca na celu ograniczenie ubóstwa i wykluczenia społecznego.
				\item Równe szanse na zadowalające życie dla wszystkich.
				\item Nakręcanie koniunktury.
				\item Więcej czasu dla rodziny (opieka nad dziećmi i rodzicami) i na podnoszenie kwalifikacji, do tego opłacanego przez państwo.
				\item Krótszy czas pracy i zwiększenie zatrudnienia.
				\item Uelastycznienie (dla pracownika) rynku pracy, jego ``ucywilizowanie'' i zmuszenie przedsiębiorców do oferowania ``godnej płacy'' (autor nie używa dokładnie takich słów).
				\item Zmniejszenie szarej strefy.
				\item Konieczność, gdyż automatyzacja pozbawi nas miejsc pracy.
				\item Likwidacja fikcji urzędniczej, polegającej na tym, że urzędnicy pomocy społecznej zajmują się pilnowaniem, żeby bezrobotni nie pracowali na czarno.
				\item Zniknięcie stygmatyzacji beneficjentów pomocy społecznej.
			\end{enumerate}
			\item Szczegółowe argumenciki na poparcie każdego argumentu z (3).
			\item Między wierszami argumenty obalające (potencjalne) kontrargumenty. Dochód podstawowy:
			\begin{enumerate}
				\item Wcale nie wpływa niekorzystnie na podaż i etykę pracy.
				\item Wcale nie powoduje uzależnienia od państwa (``od świadczenia'').
				\item Wcale nie spowoduje spadku wynagrodzeń spowodowanego tym, że pracodawcy wliczyliby sobie dochód podstawowy do oferowanej pensji.
				\item Wcale nie zniechęca do podjęcia pracy.
				\item Politycy wcale nie prześcigaliby się w rozdawnictwie.
			\end{enumerate}
		\end{enumerate}
	
	\section{Odparcie argumentów autora}
		Ogólne argumenty podane w podpunkcie (2) poprzedniej sekcji można dość łatwo odeprzeć lub wykazać ich niepełność:
		\begin{enumerate}
			\item Pojęcie ``wolności realnej'' jest chochołem, którego celem jest wprowadzenie zamętu pojęciowego oraz przejęciem kontroli nad językiem debaty. Wolność, jakiegokolwiek przymiotnika by do niej nie dostawić, z pewnością nie oznacza prawa do okradania innych (a tym jest każda forma redystrybucji).
			\item Dochód podstawowy nie ma nic wspólnego ze sprawiedliwością. Sprawiedliwością nie jest w szczególności sytuacja, w której rząd zabiera pracującemu człowiekowi pieniądze i oddaje je nierobowi, któremu nie chce się pracować.
			\item Godność, autonomia i emancypacja jednostki to puste frazesy i ładnie brzmiące hasełka, które nie mają jednak przełożenia na praktykę. Co więcej, bycie zależnym od świadczeń rządowych jest dokładnym przeciwieństwem autonomii i emancypacji.
			\item Sztuczne ograniczenie ubóstwa i wykluczenia społecznego jest niemożliwe. Zjawiska te wynikają z różnych czynników - złych genów, błędnych decyzji życiowych oraz zwykłego pecha. Nie da się uczynić wszystkich równymi (chyba, że równie biednymi). Podobnie niemożliwe jest sprawić, aby wszyscy mieli równe szanse na godne życie.
			\item Dochód podstawowy nie nakręciłby koniunktury - wprost przeciwnie, wpłynąłby on na nią bardzo niekorzystnie. Wynika to z faktu, że jego wprowadzenie wymagałoby gigantycznych pieniędzy, które rząd musiałby komuś zabrać. Taki bodziec podziałał by na okradzionych (czyli pracowników, przedsiębiorców i oszczędząjących) bardzo niekorzystnie, demotywując ich do produktywnej pracy.
			\item Dochód podstawowy z pewnością sprawiłby, że ludzie mieliby więcej czasu dla rodziny, jednak uznanie tego za coś pozytywnego jest absurdem. Gdyby ci ludzie cenili czas spędzony z rodziną wyżej niż pracę, to już teraz ograniczyliby czas poświęcany pracy i przeznaczyli go rodzinie. Argument równie absurdalny co ten przewijający się ostatnimi czasy w sprawie zakazu handlu w niedzielę.
			\item Dochód podstawowy mógłby skrócić czas pracy i tym samym zwiększyć zatrudnienie, byłoby to jednak skrajnie niekorzystne. Wynika to z faktu, że obecni dobrzy i produktywni pracownicy (a tacy oni są, jeżeli tylko firma jeszcze nie zbankrutowała) pracowaliby mniej, a na część ich etatów przedsiębiorcy musieliby dokoptować pracowników gorszych i mniej produktywnych (gdyby tacy nie byli, to już dawno zostaliby zatrudnieni).
			\item ``Ucywilizowanie'' rynku pracy oznaczałoby wzrost kosztów pracy, a zatem miałoby ten sam negatywny skutek, co np. płaca minimalna.
			\item Dochód podstawowy mógłby zmniejszyć szarą strefę, ale tylko przy założeniu, że ludzie pracują tam dlatego, że nie chcą stracić zasiłków dla bezrobotnych. Jest całkiem wyobrażalne, że po wprowadzeniu dochodu podstawowego ludzie dalej pracowaliby w szarej strefie, gdyż dzięki temu nie muszą np. płacić podatków.
			\item Co do automatyzacji, to luddyści straszą nią już od ponad dwustu lat, od samych początków rewolucji przemysłowej, i apokalipsa jeszcze nie nastąpiła. Wprost przeciwnie, dzięki wprowadzeniu maszyn i innych technologii ludzi jest obecnie kilka razy więcej niż kiedyś i wszyscy żyją na dużo wyższym poziomie.
			\item Argument o wyrwaniu się z tyranii mężów etc. jest tak głupi, że szkoda go komentować.
			\item Co do pozostałych argumentów, to trzeba przyznać, że wprowadzenie dochodu podstawowego mogłoby (choć nie jestem przekonany, czy akurat w Polsce...) znacznie uprościć system świadczeń społecznych oraz zlikwidować fikcję urzędniczą oraz biurokrację. Z pewnością zniknęłaby także stygmatyzacja beneficjentów pomocy społecznej.
		\end{enumerate}	
	
	\section{Ocena argumentacji metodą ARS} 
		\begin{itemize}
			\item Relewancja: argumentacja autora jest z pewnością relewantna, ale niekompletna - argumenty przeciw zostały pominięte lub odparte bardzo pobieżnie.
			\item Akceptowalność: tekst jest jawnie stronniczy i naszpikowany niejawnymi przesłankami, zaś argumenty autora są powierzchowne i nie widać w nich głębszej refleksji. Jest to mocnym argumentem na to, że konkluzje tekstu należy odrzucić.
			\item Wystarczalność: argumenty autora odparłem powyżej, więc nie są one wystarczające.
		\end{itemize}
		
	\section{Nieuwzględnione możliwości}
		\par Jednak nawet gdyby argumenty autora były dobre, dalej byłyby niewystarczające, gdyż autor nie rozważa wcale kosztów dochodu podstawowego ani ich wpływu na budżet. Jest to miażdżący mankament tekstu, gdyż koncepcja dochodu podstawowego ze względu na cechy powszechności i równości świadczenia jest bardzo prosta do wycenienia i ocenienia z punktu widzenia budżetu: wystarczy policzyć upoważnionych (a więc wszystkich obywateli lub wszystkich dorosłych, co jest łatwe) i pomnożyć przez kwotę świadczenia, a następnie przez 12 miesięcy.
		\par Przyznanie każdemu Polakowi 1500 złotych miesięcznie to koszt rzędu 684 mld złotych. Jego poniesienie oznaczałoby wzrost wydatków sektora publicznego o 87\% (dane za mapą wydatków państwa Fundacji Republikańskiej), a więc wprowadzenie dochodu podtawowego w takiej formie jest niemożliwe.
		\par Nawet gdyby dochód podstawowy przyznać jedynie dorosłym Polakom (na oko ok. 30 mln osób), zabrać emerytom emeryutry, zlikwidować pomoc społeczną (w tym 500+, zasiłki dla bezrobotnych i urzędy pracy) oraz przestać spłacać dług publiczny, koszty przyznania każdemu dorosłemu 1500 zł miesięcznie wyniosłyby w skali roku ok. 200 mld złotych, co przekłada się na wzrost wydatków publicznych o 25\%.
		\par Wprowadzenie dochodu podstawowego (przynajmniej w Polsce) jest zupełnie niemożliwe z powodów czysto budżetowych, nie wspominając o oporze politycznym (olbrzymia rzesza emerytów byłaby na tym stratna, w tym silne politycznie grupy jak górnicy).
		\par Jak więc widać, najbardziej rażącym brakiem tekstu jest analiza kosztów funkcjonowania dochodu podstawowego - autor wspomina wprawdziwe, że koncepcja ta ``może wydawać się nierealistyczna'', ale potem zupełnie ignoruje ten temat.

\end{document}